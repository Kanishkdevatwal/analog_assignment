\let\negmedspace\undefined
\let\negthickspace\undefined
\documentclass[journal,12pt,onecolumn]{IEEEtran}
\usepackage{cite}
\usepackage{amsmath,amssymb,amsfonts,amsthm}
\usepackage{algorithmic}
\usepackage{graphicx}
\usepackage{textcomp}
\usepackage{xcolor}
\usepackage{txfonts}
\usepackage{listings}
\usepackage{enumitem}
\usepackage{mathtools}
\usepackage{gensymb}
\usepackage{comment}
\usepackage[breaklinks=true]{hyperref}
\usepackage{tkz-euclide} 
\usepackage{listings}
\usepackage{gvv}                                        
\def\inputGnumericTable{}                                 
\usepackage[latin1]{inputenc}                                
\usepackage{color}                                            
\usepackage{array}                                            
\usepackage{longtable}                                       
\usepackage{calc}                                             
\usepackage{multirow}                                         
\usepackage{hhline}                                           
\usepackage{ifthen}                                           
\usepackage{lscape}

\newtheorem{theorem}{Theorem}[section]
\newtheorem{problem}{Problem}
\newtheorem{proposition}{Proposition}[section]
\newtheorem{lemma}{Lemma}[section]
\newtheorem{corollary}[theorem]{Corollary}
\newtheorem{example}{Example}[section]
\newtheorem{definition}[problem]{Definition}
\newcommand{\BEQA}{\begin{eqnarray}}
\newcommand{\EEQA}{\end{eqnarray}}
\newcommand{\define}{\stackrel{\triangle}{=}}
\theoremstyle{remark}
\newtheorem{rem}{Remark}
\begin{document}

\bibliographystyle{IEEEtran}
\vspace{3cm}

\title{11.15}
\author{EE23BTECH11029 - Kanishk}
\maketitle

\bigskip

\renewcommand{\thefigure}{\theenumi}
\renewcommand{\thetable}{\theenumi}
\footnotesize
\textbf{Question}:\\ 

A SONAR system fixed in a submarine operates at a frequency 40.0 kHz. An enemy submarine moves towards the SONAR with a speed of 360 km/hr. What is the frequency of sound reflected by the submarine? Take the speed of sound in water to be 1450 m/s.\\

\textbf{Solution}:\\

\begin{table}[ht]
    \centering
    \def\arraystretch{1.5}
    \begin{tabular}{|c|c|c|}
   
   \hline
   Parameter & Description & Value\\
   \hline
   $V $& Speed of sound in water & 1450m/s\\
   \hline 
   $V_e$ & Speed of enemy submarine & 100m/s\\
   \hline 
   $Vrel$ & Relative velocity between both submarine & 1550m/s\\
   \hline
   $f $& Frequency of SONAR wave & 40kHz\\ 
   \hline
   $y\brak{x,t}$ & Equation of SONAR wave & $A\sin\brak{2\pi ft-\frac{2\pi}{\lambda}x+\phi}$\\
   \hline
   $\lambda$ & Wavelength of SONAR wave & 3.625cm\\
    \hline
   $ f' $& Frequency observed by enemy submarine & 42.76kHz\\
   \hline
   $\lambda_2$ & Wavelength of reflected wave & 3.157cm\\
   \hline 
   $T=\frac{1}{f'}$ & Time period of reflected wave & 23.38s\\
   \hline
  $ y_2\brak{x,t}$ & Equation of reflected wave as observed from submarine& $A\sin\brak{2\pi f''t-\frac{2\pi}{\lambda_2}x+\phi}$\\
   \hline
   
\end{tabular}   
   \caption{Input Parameters}
   \label{tab:11.15.25}
\end{table}

Let us assume that the wave is reflected completely from enemy submarine.\\


From \tabref{tab:11.15.25} :

\begin{align}
Vrel&=V+V_e\\
f'&=Vrel/\lambda\\
&=\brak{\frac{V+Ve}{V}}f\\
&= \brak{\frac{1450+100}{1450}}40\\
\implies f' &=42.76kHz
\end{align}

\begin{align}
\lambda_2&=T\brak{V-V_e}\\
&=\brak{\frac{V-V_e}{f'}}\\
f''&=V/\lambda_2\\
\implies f'' &=\brak{\frac{V}{V-V_e}}f'\\
\therefore f''&=45.92kHz
\end{align}







\newpage
Let us assume that amplitude of both waves is $1cm$.

\begin{figure}[h]
    \centering
    \includegraphics[width=\columnwidth]{figs/fig1.png}\\
    \caption{Graph of SONAR and reflected waves}
    \label{tab:11.15.25.1}
\end{figure}

\vspace{0.1cm}
	
	$A$ : SONAR submarine

	$B$ : Enemy submarine
	
	Blue graph : SONAR wave

	Orange  graph : Reflected wave



\end{document}



  



